 \documentclass[paper=a4, fontsize=11pt]{scrartcl}
\usepackage[utf8]{inputenc}
\usepackage[T2A]{fontenc}
\usepackage{amssymb}
\usepackage{fourier}
\usepackage[english, russian]{babel}
\usepackage{amsmath,amsfonts,amsthm} % Math packages
\usepackage[pdftex]{graphicx}
\usepackage{url}

\usepackage{sectsty}
\allsectionsfont{\centering \normalfont\scshape}

%%% Custom headers/footers (fancyhdr package)
\usepackage{fancyhdr}
\pagestyle{fancyplain}
\fancyhead{}                                            % No page header
\fancyfoot[L]{}                                         % Empty 
\fancyfoot[C]{}                                         % Empty
\fancyfoot[R]{\thepage}                                 % Pagenumbering
\renewcommand{\headrulewidth}{0pt}          % Remove header underlines
\renewcommand{\footrulewidth}{0pt}              % Remove footer underlines
\setlength{\headheight}{13.6pt}

\title{
        \huge Долги \\
}
\date{}
\author{}
 
\begin{document}
\maketitle

1. Дополнительные материалы.

2. Куча важных тудушек к 4 неделе.

3. Перенести задачи с тестов и экзаменов в более общие разделы.

4. Найдите количество циклических последовательностей длины 16 из трёх символов, в которых встречаются ровно два символа из трёх.
Incorrect

5. Пусть N --- количество способов разменять купюру в 100 рублей на монеты достоинством 1, 2 и 5 рублей. Напишите в ответе ближайшее к N снизу целое число, которое делится на 100.
\end{document}

